\chapter{Preliminary Analysis}\label{Chap:2}

\section{Challenges}

There are various challenges in tackling the problem, namely, which features to use, and which machine learning technique to use. There might also be a possible lack of direct match between two teams, where we will need to estimate the result.

%\subsection{Feature Selection}
%
%Features are pieces of information that may be useful for predictions. As mentioned in Section~\ref{Sec:data_set}, Kaggle provides a comprehensive set of various data which might be useful for us. However, we have to choose the features that satisfy the two criteria of usefulness and relevance to tackle this problem.

\subsection{Obsolete Data Sets}

Regardless of the machine-learning technique used, the most \emph{relevant} data is the historical match records during both regular and tournament seasons. However, some data sets contain match records that date back to as early as 1985, which are no longer \emph{useful} in today's context. After all, the NCAA tournament teams consist of \emph{college students} who can only stay with a team for a maximum of four years before graduation. Since only team-level data is available, it is difficult to quantify the effect that changes in a team's composition bring on the skill of that team, as players constantly leave and join the team. Moreover, there were some substantial updates on the Tournament's rules at the beginning of the 2008-2009 season~\cite{NP15}, which also affected the strategy teams used in the subsequent tournaments. Therefore, only the most recent match records are useful for prediction. For the purpose of this project, a four-year window from 2013 to 2016 is selected and all data used in this project fall within this window. 

\subsection{Complex Interrelationship}

No team is in isolation. The tournament matches are interactive processes whereby complex interrelationships exist amongst all the contesting teams, which adds another layer of complication in the attempt to predict match results. For example, given the history match records that Team A beaten Team B and Team C lost to Team B, one should intuitively conclude that Team A should have a higher probability of beating Team C. But what if another record shows that Team A once lost to Team C? In that case, the relative strength levels of the three teams will be hard to determine. Moreover, the match whose result is to be predicted may be the very \emph{first} match ever between two teams. In other words, there are no historical records that give a direct assessment on the two teams' strengths. Such lack of knowledge must be complemented by some interrelationships amongst the teams. So a good predictive model should not only be able to take into consideration the current game record, but also explore the interrelationships amongst all the game records and generalise on unseen matches. 

\subsection{The Curse of Model Popularity}

